% frktest.tex v1.2 <1998-10-03>

\documentclass[a4paper]{article}
\oddsidemargin=-1cm
\textwidth=15cm
\usepackage{german}
\usepackage{yfonts}[1998/10/03]

\pagestyle{empty}

\begin{document}

%%%%%%%%%%%%%%%%%%%%%%%%%%%%%%%%%%%%%%%%%%%%%%%%%%%%%%%%%%%%%%%%%%%%%%

% a short nonsense text to show the gothic font; 
% note the small \baselineskip of 10pt with a font size of 14.4pt:

\begin{minipage}{10cm}
\gothfamily\Large\baselineskip10pt
Nonummy sid semper aliena auditorum lorem ipsum quod omnia sunt 
peccatores:: Lorem ipsum dolor sit amet con sectetuer adipiscing elit:
Sed diam tempor incidunt ut labore et dolor magna aliquam erat volupat
ut wisi enim ad minim veniam quis: nostrud oblationem
corporemque suscipit laboris:

\end{minipage}
\vspace{2.5cm}

%%%%%%%%%%%%%%%%%%%%%%%%%%%%%%%%%%%%%%%%%%%%%%%%%%%%%%%%%%%%%%%%%%%%%%

% a paragraph in Fraktur, with Schwabacher used for emphasizing;
% note the small e on the umlauts:

\begin{minipage}{14cm}
\parindent=1cm % watch it being ignored!
\frakfamily\Large\fraklines
\yinipar{D}ie \textswab{Orgel}, der \textswab{Fl*ugel},
das: \textswab{Fortepiano} und das: \textswab{Clavicord} sind die
gebr*auchlichsten Clavierinstrumente zum Accompagnement. Es: ist Schade,
da"s die sch*one Erfindung des: \textswab{Holfeldischen Bogenclaviers:}
noch nicht gemeinn*utzig geworden ist; man kann dahero dessen besondere
Vorz*uge hierinnen noch nicht genau bestimmen. Es: ist gewi"s zu 
glauben, da"s es: sich auch bey der Begleitung gut aus:nehmen werde.

% Aus/from:
% Bach, Carl Philipp Emanuel: Versuch "uber die wahre
% Art das Clavier zu spielen. Zweyter Theil, in welchem die Lehre von
% dem Accompagnement und der freyen Fantasie abgehandelt wird.
% G. L. Winter, Berlin 1762.

\end{minipage}
\vspace{2.5cm}

%%%%%%%%%%%%%%%%%%%%%%%%%%%%%%%%%%%%%%%%%%%%%%%%%%%%%%%%%%%%%%%%%%%%%%

% another paragraph in Fraktur, with a Schwabacher heading; 
% we use regular umlauts here:

\begin{minipage}{9.5cm}
\large\fraklines
\centerline{{\Large\swabfamily Der Orakelbrunnen bei St. Moritz}}%
\vspace{.5cm}

\frakfamily
\yinipar{V}on dem stattlichen Orte Kirchehrenbach in der Fr"ankischen
Schweiz f"uhrt ein einsames:  Str"a"schen, an der steil ab\-st"ur\-zen\-den
Breitseite der Ehrenb"urg vor"uber, gen Leutenbach.  In einer halben
Stunde ist das:  schmucke Pfarrdorf erreicht.  S"ud"ostlich des:selben,
kaum eine Viertelstunde entfernt, treffen wir auf dem Wege nach 
Ortspitz
in einem Seitent"alchen von unber"uhrter Natursch"onheit das:  uralte,
einsame Kirchlein "`Sankt Moritz"' mit seinem stillen Friedhofe.  In der
N"ahe des:  Kirchleins:  steht ein niedriges:, "uber eine Quelle
erbautes:  Feld\-kapellchen mit der Statue des:  hl.~Mauritius:,
dargestellt als:  Ritter in voller Wehr.  
\par
\end{minipage} 

% Aus/from:
% Br"uckner, Karl:
% Am Sagenborn der Fr"ankischen Schweiz.
% Frankenverlg v. Gg. Kohler, Wunsiedel 1921.


\end{document}

% finis
